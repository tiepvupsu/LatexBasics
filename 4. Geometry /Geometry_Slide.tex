% \documentclass[xcolor=dvipsnames,10pt]{beamer}
\documentclass[10pt,xcolor=dvipsnames]{beamer}

% \documentclass{beamer}

\usepackage[utf8]{inputenc} 
\usepackage[T1]{fontenc}
% \usepackage[frenchb]{babel}
\usepackage{setspace}
\usepackage{xcolor}
\usepackage{listings}
\lstset
{
    language=[LaTeX]TeX,
    breaklines=true,
    basicstyle=\tt\scriptsize,
    keywordstyle=\color{black},
    identifierstyle=\color{black}, 
}

\graphicspath{ {tikz/} }

% \usepackage{beamerthemebars}
% \usepackage[bars]{beamerthemetree}
\usepackage{tkz-euclide}
\usetkzobj{all} 

\usepackage{multicol}

\usepackage{appendixnumberbeamer}
 \usepackage[framed,autolinebreaks,useliterate]{mcode}
\usepackage{listings}
% \setbeamercovered{highly dynamic}
% \input{mypackagebeamer}
% \input{definitions}

\usepackage{tikz}
\usetheme{Dresden}
% \usetheme{default}
%\usecolortheme{lily}
\useoutertheme[subsection=false]{smoothbars}
%\useoutertheme[subsection=false]{miniframes}
\useinnertheme{circles}
\usefonttheme[onlymath]{serif}
\setbeamertemplate{navigation symbols}
%\setRL
%\insertslidenavigationsymbol

\setbeamertemplate{footline}[frame number]


\setbeamersize{text margin left=0.5cm,text margin right=0.5cm}
\setbeamersize{text margin left=0.5cm}
% \usepackage{beamerthemebars}
% \usepackage[bars]{beamerthemetree}
% \setbeamercovered{higly dynamic}
%% ================== block: Information ==========================
% \setbeamertemplate{footline}[body]
% \title[]{DFDL: \\Discriminative Feature-oriented \\Dictionary Learning \\for Histopathological Image Classification}
% \subtitle{}
% \author[]{\small Tiep Huu Vu\\ \tcb{Master Paper Presentation}\\
% \vspace{3mm} \includegraphics[scale=0.3]{figs/ipal.png}  }

% \date[]{
% \vspace{3mm} April 9, 2015}
% \institute[]{Department of Electrical Engineering \\  \vspace{2mm}Pennsylvania State University }
%% ------------------end of block: Information ----------------------------

\def\changemargin#1#2{\list{}{\rightmargin#2\leftmargin#1}\item[]}
\let\endchangemargin=\endlist 


\setbeamertemplate{navigation symbols}{}

%% ================== commands ==========================
\newcommand{\myShowPoints}[2]{
\tkzDrawPoints(#1) \tkzLabelPoints[#2](#1)
}   

\newcommand{\myGetMidPoint}[3]{
\tkzDefMidPoint(#1,#2)\tkzGetPoint{#3}
}   
%% ------------------ end of commands -------------------
  %%%%Title Slide{
\title[Discriminative Dictionary Learning]{\LaTeX ~basics}
\subtitle{}
\author[Tiep Huu Vu]{Drawing geometric objects with \href{http://www.altermundus.fr/downloads/documents/Sangaku.pdf}{tkz-euclide} package}
\titlegraphic{\includegraphics[height=4.5cm]{triangle_euler.pdf}} % instead of \logo 
\institute {\\
\par Tiep Vu Huu\\ \vspace{4mm}}

\begin{document}
%----------- titlepage ----------------------------------------------%
% \begin{frame}[plain]
%   \titlepage 
% \end{frame}

{
\usebackgroundtemplate{\includegraphics[width=\paperwidth]{background_geometry.pdf}}%
\begin{frame}[plain]
\end{frame}
}

\frame
{
  \frametitle{Outline}
  \tableofcontents
}



% subsection defining_points (end)

% section points (end)
% =========================== New Slide =================================
\begin{frame}[fragile]
\frametitle{tkz-euclide}

\begin{columns}
\begin{column}{0.5\textwidth}
\begin{itemize}

\item documentclass
\vspace{-.2in}

  \begin{lstlisting}
\documentclass{standalone} 
%\documentclass{article}
\end{lstlisting}

\item package
\vspace{-.2in}
\begin{lstlisting}
\usepackage{tkz-euclide}
\usetkzobj{all}
\end{lstlisting}

\item main body
\vspace{-.2in}
\begin{lstlisting}
\begin{document}
% \begin{figure}
\begin{tikzpicture}
   ....
\end{tikzpicture}
% \end{figure}
\end{document}
  \end{lstlisting}
\end{itemize}
\href{http://www.altermundus.fr/downloads/documents/Sangaku.pdf}{\color{blue} See tkz-euclide examples here.}
\end{column}
\begin{column}{0.5\textwidth}  %%<--- here
  \centering
  \includegraphics{point1.pdf}
\end{column}
\end{columns}
\end{frame}
% =========================== New Slide =================================
% \begin{frame}[fragile]
% \frametitle{Points}

% \begin{columns}
% \begin{column}{0.6\textwidth}
%   \begin{lstlisting}
% %% point1.tex
% ...
% \begin{document}
% \begin{tikzpicture}
%    % defining 1 point
%    \tkzDefPoint(-3,0){A}
%    % defining multiple points 
%    \tkzDefPoints{1/1/B, -2/2/C, 1/-2/D}
%    % drawing points as dots
%    \tkzDrawPoints(A,B)
%    % labeling points
%    \tkzLabelPoints[above](A,B)
%    \tkzLabelPoints[right](C,D)
% \end{tikzpicture}
% \end{document}
%   \end{lstlisting}
% \end{column}
% \begin{column}{0.4\textwidth}  %%<--- here
%   \centering
%   \includegraphics{point1.pdf}
% \end{column}
% \end{columns}
% \end{frame}

% =========================== New Slide =================================
\section{Points} % (fold)
\label{sec:points}

\begin{frame}
  \frametitle{}
\begin{center}
\Large{{\color{black}  Learn \LaTeX \\Drawing geometric objects with tkz-euclide package}}\\
\Huge{{\color{blue}  Points}}\\
\vspace{.1in}
\includegraphics[scale = .7]{relative_coordinate.pdf}
\par \large{{\color{black}  Tiep Vu Huu}}
\end{center}
\end{frame}

% =========================== New Slide =================================
\subsection{Points in Cartesian coordinate system} % (fold)
\label{sub:defining_points}

\begin{frame}[fragile]
\frametitle{Points in Cartesian coordinate system}

\begin{columns}
\begin{column}{0.5\textwidth}
  \begin{lstlisting}
%% def_point_Cartesian.tex
...
\begin{document}
\begin{tikzpicture}
   ...
   % Cartesian coordinate (x,y)
   \coordinate (O) at (0, 0);
   \coordinate (A) at (1, 2);
   % using tkz-euclide
   \tkzDefPoint(1,1){C}
   \tkzDefPoints{-1/2/D, 2/-3/E}
   ...
\end{tikzpicture}
\end{document}
  \end{lstlisting}
  \href{http://www.highschoolmathandchess.com/latex/altermundus-packages/points-lines-line-segments-rays-and-labels/}{\color{blue} See more here.}
\end{column}
\begin{column}{0.5\textwidth}  %%<--- here
  \centering
  \includegraphics[scale = .7]{def_point_Cartesian.pdf}
\end{column}

\end{columns}
\end{frame}


% =========================== New Slide =================================
% subsection polar_coordinate_system (end)
\begin{frame}[fragile]
\frametitle{Drawing and Labeling points}

\begin{columns}
\begin{column}{0.65\textwidth}
  \small{\begin{lstlisting}
%% draw_label_points.tex
...
\begin{tikzpicture}
  ...
  % drawing points
  \tkzDrawPoints(O)
  \tkzDrawPoints[size=3, fill=black](A)
  \tkzDrawPoints[size=3, fill=blue](D,E)
  % labeling points
  \tkzLabelPoints(O,A,D)
  \tkzLabelPoints[red](E)
  % labeling in mathmode
  \tkzDefPoint[label = below:$A_2$](1,1){A2}
  \tkzDrawPoints[size=3, fill=blue](A2)
\end{tikzpicture}
    \end{lstlisting}}
\end{column}
\begin{column}{0.4\textwidth}  %%<--- here
  \centering
  \includegraphics[scale = 1]{draw_label_points.pdf}
\end{column}
\end{columns}
\end{frame}
% =========================== New Slide =================================
\subsection{Polar coordinate system} % (fold)
\label{sub:polar_coordinate_system}

% subsection polar_coordinate_system (end)
\begin{frame}[fragile]
\frametitle{Points in polar coordinate system}

\begin{columns}
\begin{column}{0.5\textwidth}
  \begin{lstlisting}
%% def_point_polar.tex
...
\begin{document}
\begin{tikzpicture}
   ...
   \coordinate (O) at (0, 0);
   % polar coordinate, (theta:r)
   \coordinate (A) at (30:2);
   \coordinate (B) at (-60:1);

   \tkzDefPoint(120:1.5){C}
   ...
\end{tikzpicture}
\end{document}
  \end{lstlisting}
\end{column}
\begin{column}{0.5\textwidth}  %%<--- here
  \centering
  \includegraphics[scale = .7]{def_point_polar.pdf}
\end{column}
\end{columns}
\end{frame}

% =========================== New Slide =================================
\subsection{Relative coordinates} % (fold)
\label{sub:polar_coordinate_system}

% subsection polar_coordinate_system (end)
\begin{frame}[fragile]
\frametitle{Relative coordinates}

\begin{columns}
\begin{column}{0.6\textwidth}
  \begin{lstlisting}
%% relative_coordinate.tex
% ...
\usepackage{calc}
\begin{document}
\begin{tikzpicture}
   % ...
   \coordinate (O) at (0, 0);
   \coordinate (A) at (1, 2);
   \coordinate (B) at ($(A) + (1, 1)$);
   \coordinate (C) at ($(B) + (120:3)$);
   \coordinate (E) at ($(C)+(B)-(A)$);
   % ...
\end{tikzpicture}
\end{document}
  \end{lstlisting}
\end{column}
\begin{column}{0.4\textwidth}  %%<--- here
  \centering
  \includegraphics[scale = .7]{relative_coordinate.pdf}
\end{column}
\end{columns}
\end{frame}

% % =========================== New Slide =================================
% \begin{frame}[fragile]
% \frametitle{Points with customized macros}

% \begin{columns}
% \begin{column}{0.6\textwidth}
% \begin{itemize}
% \item newcommands
% \end{itemize}
% \vspace{-.2in}
% \begin{lstlisting}
% \newcommand{\myShowPoints}[2]{
% \tkzDrawPoints(#1) 
% \tkzLabelPoints[#2](#1)        
% \end{lstlisting}

% \begin{itemize}
% \item main
% \vspace{-.2in}
% \hspace{-.5in}
% \end{itemize}  
%   \begin{lstlisting}
% %% point1_2.tex
% ...
% %%  newcommands 
% }       
% \begin{document}
% \begin{tikzpicture}
%    \tkzDefPoint(-3,0){A}
%    \tkzDefPoints{1/1/B, -2/2/C, 1/-2/D}
%    % \tkzDrawPoints(A,B,C,D)
%    % \tkzLabelPoints[above](A,B)
%    % \tkzLabelPoints[right](C,D)
%    \myShowPoints{A,B}{left}
%    \myShowPoints{C,D}{below}
% \end{tikzpicture}
% \end{document}
%   \end{lstlisting}
% \end{column}
% \begin{column}{0.4\textwidth}  %%<--- here
%   \begin{figure}[h]
%   \centering
%     \includegraphics{point1_2.pdf}
%   \end{figure}
% \end{column}
% \end{columns}
% \end{frame}
%% ==============================================================
\section{Lines, Segments, Rays} % (fold)
\label{sec:Lines, Segments, Rays}

\begin{frame}
  \frametitle{}
\begin{center}
\Large{{\color{black}  Learn \LaTeX \\Drawing geometric objects with tkz-euclide package}}\\
\Huge{{\color{blue}  Lines, Segments, Rays}}\\
% \vspace{1in}
\includegraphics{ratio_point.pdf}
\par \large{{\color{black}  Tiep Vu Huu}}
\end{center}
\end{frame}

% =========================== New Slide =================================
\subsection{Connect points} % (fold)
\label{sub:connect_points}

% subsection connect_points (end)
\begin{frame}[fragile]
\frametitle{Line, Segment, Ray, connecting two points}
\begin{columns}
\begin{column}{0.6\textwidth}
  \begin{lstlisting}
%% line_segment_ray.tex
\begin{tikzpicture}
   % ...
   % drawing red segments
   \tkzDrawSegments[red, thick](A,B C,D)
   % drawing dashed green segment
   \tkzDrawSegments[green,dashed](A,D)
   % drawing blue line
   \tkzDrawLines[draw=blue](A,C B,D)
   % array with arrow
   \tkzDrawLines[add = 0 and 0.3, draw=violet, arrows=->](C,B)
   % ...
\end{tikzpicture}
\end{lstlisting}
\href{http://www.highschoolmathandchess.com/latex/altermundus-packages/points-lines-line-segments-rays-and-labels/}{\color{blue} See more here.}
\end{column}
\begin{column}{0.4\textwidth}  %%<--- here
  \begin{figure}[h]
  \centering
    \includegraphics{line_segment_ray.pdf}
  \end{figure}
\end{column}
\end{columns}
\end{frame}

% =========================== New Slide =================================
% \subsection{Labeling and Markers } % (fold)
% \label{sub:labeling_and_markers_}

% subsection labeling_and_markers_ (end)
\begin{frame}[fragile]
\frametitle{Labeling and Markers}
\begin{columns}
\begin{column}{0.6\textwidth}
  \begin{lstlisting}
%% lsr_label_marker.tex
\begin{tikzpicture}
   % ...
   % labeling
   \tkzLabelSegment[above=1pt, rotate=65](A,C){3 cm}
   \tkzLabelSegment[above=0pt, rotate=-25](A,D){6 cm}
   % markers
   \tkzMarkSegment[mark=|](A,C)
   \tkzMarkSegment[mark=||](A,B)
   \tkzMarkSegment[mark=|||, size = 2](C,D)
   \tkzMarkSegment[mark=x](B,D) 
   % ...
\end{tikzpicture}
\end{lstlisting}
\end{column}
\begin{column}{0.4\textwidth}  %%<--- here
  \begin{figure}[h]
  \centering
    \includegraphics{lsr_label_marker.pdf}
  \end{figure}
\end{column}
\end{columns}
\end{frame}
% =========================== New Slide =================================
% \subsection{Ratio points} % (fold)
% \label{sub:ratio_points}

% subsection ratio_points (end)
\begin{frame}[fragile]
\frametitle{Define a point between two points for a given length ratio}
\begin{columns}
\begin{column}{0.6\textwidth}
  \begin{lstlisting}
%% ratio_point.tex
\begin{tikzpicture}
   % ...
   % middle points 
   \tkzDefMidPoint(A,C)\tkzGetPoint{M}    
   % others 
   \coordinate (N) at ($(A)!1/3!(D)$);
   \coordinate (P) at ($(A)!2/3!(D)$);
   \coordinate (Q) at ($(C)!-1/2!(B)$);
   \coordinate (R) at ($(C)!1/2!(B)$);
   % ...
\end{tikzpicture}
\end{lstlisting}
\end{column}
\begin{column}{0.4\textwidth}  %%<--- here
  \begin{figure}[h]
  \centering
    \includegraphics{ratio_point.pdf}
  \end{figure}
\end{column}
\end{columns}
\end{frame}
% =========================== New Slide =================================
% \begin{frame}[fragile]
% \frametitle{Middle point and markers}
% \begin{columns}
% \begin{column}{0.6\textwidth}
%   \begin{lstlisting}
% %% midpoint_marker.tex
% \begin{tikzpicture}
%    %% defining points and lines 
%    % middle point 
%    \tkzDefMidPoint(A,C)\tkzGetPoint{M}
%    % mark segment
%    \tkzMarkSegment[mark=|](A,M)
%    \tkzMarkSegment[mark=|](C,M)
%    \tkzMarkSegment[mark=||](A,B)
%    \tkzMarkSegment[mark=x](B,D)
%    %% showing points
% \end{tikzpicture}
% \end{lstlisting}
% \end{column}
% \begin{column}{0.4\textwidth}  %%<--- here
%   \begin{figure}[h]
%   \centering
%     \includegraphics{midpoint_marker.pdf}
%   \end{figure}
% \end{column}
% \end{columns}
% \end{frame}



% =========================== New Slide =================================
\subsection{Intersections} % (fold)
\label{sub:intersections}
\begin{frame}[fragile]
\frametitle{Intersection of two lines}
\begin{columns}
\begin{column}{0.6\textwidth}
  \begin{lstlisting}
%% intersection1.tex
...
\begin{tikzpicture}
   %% defining points 
   % InterLL for intersection of `Line' and `Line'
   \tkzInterLL(A,B)(C,D) \tkzGetPoint{I}
   \tkzInterLL(A,C)(B,D) \tkzGetPoint{J}
   %% showing points
\end{tikzpicture}
\end{lstlisting}
\end{column}
\begin{column}{0.4\textwidth}  %%<--- here
  \begin{figure}[h]
  \centering
    \includegraphics{intersection1.pdf}
  \end{figure}
\end{column}
\end{columns}
\end{frame}


% subsection intersections (end)


\subsection{Orthogonal and Parallel} % (fold)
\label{sub:Orthogonal and Parallel}
% =========================== New Slide =================================
\begin{frame}[fragile]
\frametitle{Orthogonal and Parallel lines}
\begin{columns}
\begin{column}{0.65\textwidth}
  \begin{lstlisting}
%% orthogonal_parallel.tex
...
\begin{tikzpicture}
   % ...
   %% orthogonal
   \tkzDefPointBy[projection=onto C--D](A) 
   \tkzGetPoint{H}
   \tkzDefLine[orthogonal=through B](C,D) 
   \tkzGetPoint{K}
   %% parallel 
   \tkzDefLine[parallel=through B](C,D)
   \tkzDrawLine[draw = red, add = .5 and -.6](B,tkzPointResult)
   % ...
\end{tikzpicture}
\end{lstlisting}
\end{column}
\begin{column}{0.3\textwidth}  %%<--- here
  \begin{figure}[h]
  \centering
    \includegraphics[scale = .8]{orthogonal_parallel.pdf}
  \end{figure}
\end{column}
\end{columns}
\end{frame}


%% ==============================================================
\section{Angles} % (fold)
\label{sec:Angles}

\begin{frame}
  \frametitle{}
\begin{center}
\Large{{\color{black}  Learn \LaTeX \\Drawing geometric objects with tkz-euclide package}}\\
\Huge{{\color{blue}  Angles}}\\
% \vspace{1in}
\includegraphics{angle.pdf}
\par \large{{\color{black}  Tiep Vu Huu}}
\end{center}
\end{frame}
% =========================== New Slide =================================
\subsection{Specifying angle} % (fold)
\label{sub:specifying_angle}

% subsection specifying_angle (end)
\begin{frame}[fragile]
\frametitle{Angles}
\begin{columns}
\begin{column}{0.6\textwidth}
  \begin{lstlisting}
%% angle.tex
...
\begin{tikzpicture}
   % ...
  \coordinate (C) at ($(A)!1.5!30:(B)$);
  \coordinate (D) at ($(C)!.7!-60:(A)$);
   % ...
\end{tikzpicture}
\end{lstlisting}
\end{column}
\begin{column}{0.4\textwidth}  %%<--- here
  \begin{figure}[h]
  \centering
    \includegraphics{angle.pdf}
  \end{figure}
\end{column}
\end{columns}
\end{frame}
% =========================== New Slide =================================
\subsection{Labeling and Markers} % (fold)
\label{sub:specifying_angle}

% subsection specifying_angle (end)
\begin{frame}[fragile]
\frametitle{Labeling and Markers}
\begin{columns}
\begin{column}{0.6\textwidth}
  \begin{lstlisting}
%% angle_label_marks.tex
...
\begin{tikzpicture}
   % ...
   % marker
   \tkzMarkAngle[size = .5, fill = red!30](B,A,C)
   \tkzMarkAngle[size = .5, mark = ||,mksize=2](A,C,H)
   % labeling
   \tkzLabelAngle[pos=.8](B,A,C){ $60^\circ$}
   \tkzLabelAngle[pos=1.2,scale = .6](A,C,H){$30^\circ$}
   % right angle marker
   \tkzMarkRightAngle[draw =blue](A,H,C)
   % ...
\end{tikzpicture}
\end{lstlisting}
\end{column}
\begin{column}{0.4\textwidth}  %%<--- here
  \begin{figure}[h]
  \centering
    \includegraphics[scale=.9]{angle_label_marks.pdf}
  \end{figure}
\end{column}
\end{columns}
\end{frame}
% =========================== New Slide =================================
\subsection{Angle bisector} % (fold)
\label{sub:specifying_angle}

% subsection specifying_angle (end)
\begin{frame}[fragile]
\frametitle{Angle Bisector}
\vspace{-.3in}
\begin{columns}
\begin{column}{0.6\textwidth}
  \begin{lstlisting}
%% angle_bisector.tex
...
\begin{tikzpicture}
   % ...
  \tkzDefLine[bisector](C,B,A) 
  \tkzGetPoint{i}
  
  \tkzDrawBisector[draw=blue](C,A,B) 
  \tkzGetPoint{J}
   % ...
\end{tikzpicture}
\end{lstlisting}
\end{column}
\begin{column}{0.4\textwidth}  %%<--- here
  \begin{figure}[h]
  \centering
    \includegraphics[scale=.9]{angle_bisector.pdf}
  \end{figure}
\end{column}
\end{columns}
\end{frame}

%% ==============================================================
\section{Circles} % (fold)
\label{sec:Circles}

% subsection subsection_name (end)
\begin{frame}
  \frametitle{}
\begin{center}
\Large{{\color{black}  Learn \LaTeX \\Drawing geometric objects with tkz-euclide package}}\\
\Huge{{\color{blue} Circles}}\\
\vspace{.1in}
\includegraphics[scale=.9]{circle_1.pdf}
\par \large{{\color{black}  Tiep Vu Huu}}
\end{center}
\end{frame}
%% ==============New Frame ================================================
\subsection{Drawing Circles} % (fold)
\label{sub:specifying_angle}

% subsection specifying_angle (end)
\begin{frame}[fragile]
\frametitle{Drawing Circles}
\vspace{-.3in}
\begin{columns}
\begin{column}{0.6\textwidth}
  \begin{lstlisting}
%% circle_1.tex
...
\begin{tikzpicture}
   % ...
   % center A, passing B
   \tkzDrawCircle[draw = red](A,B)
   % diameter BC
   \tkzDrawCircle[diameter, draw = green](B,C)
   % center B, radius 1 cm
   \tkzDrawCircle[R, draw = blue](B, 1 cm)
   % passing A, B, C 
   \tkzDrawCircle[circum](A,B,C)
   % get its center
   \tkzCircumCenter(A,B,C)\tkzGetPoint{O}
   % ...
\end{tikzpicture}
\end{lstlisting}
\href{http://www.highschoolmathandchess.com/latex/altermundus-packages/circles/}{\color{blue}See More Examples here.}
\end{column}
\begin{column}{0.4\textwidth}  %%<--- here
  \begin{figure}[h]
  \centering
    \includegraphics[scale=.9]{circle_1.pdf}
  \end{figure}
\end{column}
\end{columns}
\end{frame}

%% ==============New Frame ================================================
\subsection{Circle Intersection} % (fold)
\label{sub:specifying_angle}

% subsection specifying_angle (end)
\begin{frame}[fragile]
\frametitle{Circle Intersection}
\vspace{-.3in}
\begin{columns}
\begin{column}{0.6\textwidth}
  \begin{lstlisting}
%% circle_intersection.tex
...
\begin{tikzpicture}
   % ...
   % Line-Circle intersection
   \tkzInterLC(C,D)(A,B) 
   \tkzGetPoints{E}{F}
   % Circle-Circle intersection
   \tkzInterCC(A,B)(I,J) 
   \tkzGetPoints{G}{H}
   % ...
\end{tikzpicture}
\end{lstlisting}
\href{http://www.highschoolmathandchess.com/latex/altermundus-packages/circles/}{\color{blue}See More Examples here.}
\end{column}
\begin{column}{0.4\textwidth}  %%<--- here
  \begin{figure}[h]
  \centering
    \includegraphics[scale=.9]{circle_intersection.pdf}
  \end{figure}
\end{column}
\end{columns}
\end{frame}

%% ==============New Frame ================================================
\subsection{Circle and Tangents} % (fold)
\label{sub:specifying_angle}

% subsection specifying_angle (end)
\begin{frame}[fragile]
\frametitle{Circle and Tangents}
\vspace{-.3in}
\begin{columns}
\begin{column}{0.6\textwidth}
  \begin{lstlisting}
%% circle_tangent.tex
...
\begin{tikzpicture}
   % ...
   \tkzDrawCircle[draw = red](A,B)
   % from a point on the circle
   \tkzTangent[at=B](A) \tkzGetPoint{h}
   % from a point outside the circle
   \tkzTangent[from=C](A,B) 
   \tkzGetPoints{M}{N}
   % ...
\end{tikzpicture}
\end{lstlisting}
\href{http://www.highschoolmathandchess.com/latex/altermundus-packages/circles/}{\color{blue}See More Examples here.}
\end{column}
\begin{column}{0.4\textwidth}  %%<--- here
  \begin{figure}[h]
  \centering
    \includegraphics[scale=.8]{circle_tangent.pdf}
  \end{figure}
\end{column}
\end{columns}
\end{frame}
% section circle (end)

%% ==============================================================
\section{Triangles} % (fold)
\label{sec:Triangles}

% subsection subsection_name (end)
\begin{frame}
  \frametitle{}
\begin{center}
\Large{{\color{black}  Learn \LaTeX \\Drawing geometric objects with tkz-euclide package}}\\
\Huge{{\color{blue} Triangles}}\\
\vspace{.1in}
\includegraphics[scale=.6]{triangle_euler.pdf}
\par \large{{\color{black}  Tiep Vu Huu}}
\end{center}
\end{frame}

\subsection{Drawing triangles} % (fold)
\label{sub:drawing_triangles}
%% ==============New Frame ================================================
% subsection specifying_angle (end)
\begin{frame}[fragile]
\frametitle{Drawing triangles}
\vspace{-.3in}
\begin{columns}
\begin{column}{0.6\textwidth}
  \begin{lstlisting}
%% triangle_1.tex
...
\begin{tikzpicture}
   % ...
   % connecting 3 points
   \tkzDrawPolygon(A,B,C)
   % equilateral triangles
   \tkzDefTriangle[equilateral](C,B) 
   \tkzGetPoint{D}
   \tkzDrawPolygon[fill=green!30](B,C,D)
   % ...
\end{tikzpicture}
\end{lstlisting}
\end{column}
\begin{column}{0.4\textwidth}  %%<--- here
  \begin{figure}[h]
  \centering
    \includegraphics[scale=.8]{triangle_1.pdf}
  \end{figure}
\end{column}
\end{columns}
\end{frame}

% subsection specifying_angle (end)
%% ==============New Frame ================================================
\subsection{Centroid, Orthocenter, Circumcircle, Inscribed Circle} % (fold)
\label{sub:centroid_orthocenter_circumcircle_inscribed_circle}

% subsection centroid_orthocenter_circumcircle_inscribed_circle (end)
\begin{frame}[fragile]
\frametitle{Centroid}
\vspace{-.3in}
\begin{columns}
\begin{column}{0.6\textwidth}
  \begin{lstlisting}
%% triangle_centroid.tex
...
\begin{tikzpicture}
   % ...
   % get centroid
   \tkzCentroid(A,B,C)\tkzGetPoint{G}
   % drawing median lines
   \tkzDrawLines[add = 0 and 1/2](A,G B,G C,G)
   % ...
\end{tikzpicture}
\end{lstlisting}
\end{column}
\begin{column}{0.4\textwidth}  %%<--- here
  \begin{figure}[h]
  \centering
    \includegraphics[scale=.8]{triangle_centroid.pdf}
  \end{figure}
\end{column}
\end{columns}
\end{frame}

%% ==============New Frame ================================================
\begin{frame}[fragile]
\frametitle{Orthocenter}
% \vspace{-.3in}
\begin{columns}
\begin{column}{0.6\textwidth}
  \begin{lstlisting}
%% triangle_orthocenter.tex
...
\begin{tikzpicture}
   % ...
   % drawing altitudes 
  \tkzDrawAltitude[draw =blue](B,C)(A) \tkzGetPoint{D}
  \tkzDrawAltitude[draw =blue](A,C)(B) \tkzGetPoint{E}
  \tkzDrawAltitude[draw =blue](B,A)(C) \tkzGetPoint{F}
  % get the orthocenter 
  \tkzInterLL(A,D)(B,E) \tkzGetPoint{H}
   % ...
\end{tikzpicture}
\end{lstlisting}
\end{column}
\begin{column}{0.4\textwidth}  %%<--- here
  \begin{figure}[h]
  \centering
    \includegraphics[scale=.5]{triangle_orthocenter.pdf}
  \end{figure}
\end{column}
\end{columns}
\end{frame}

%% ==============New Frame ================================================
\begin{frame}[fragile]
\frametitle{Circumcircle}
% \vspace{-.3in}
\begin{columns}
\begin{column}{0.6\textwidth}
  \begin{lstlisting}
%% triangle_circumcircle.tex
...
\begin{tikzpicture}
   % ...
  % draw the circumcircle
  \tkzDrawCircle[circum](A,B,C)
  % get its center 
  \tkzCircumCenter(A,B,C)\tkzGetPoint{O}
  % draw perpendicular bisector lines 
  \tkzDrawAltitude[draw =blue](B,C)(O) \tkzGetPoint{D}
  \tkzDrawAltitude[draw =blue](A,C)(O) \tkzGetPoint{E}
  \tkzDrawAltitude[draw =blue](B,A)(O) \tkzGetPoint{F}
   % ...
\end{tikzpicture}
\end{lstlisting}
\end{column}
\begin{column}{0.4\textwidth}  %%<--- here
  \begin{figure}[h]
  \centering
    \includegraphics[scale=1]{triangle_circumcircle.pdf}
  \end{figure}
\end{column}
\end{columns}
\end{frame}

%% ==============New Frame ================================================
\begin{frame}[fragile]
\frametitle{Inscribed circle}
% \vspace{-.3in}
\begin{columns}
\begin{column}{0.6\textwidth}
  \begin{lstlisting}
%% triangle_inscribedcircle.tex
...
\begin{tikzpicture}
   % ...
  % get the inscbided center 
  \tkzInCenter(A,B,C) \tkzGetPoint{I}
  % project it into one edge
  \tkzDrawAltitude(B,C)(I) \tkzGetPoint{H}
  % draw the circle 
  \tkzDrawCircle[draw = red](I,H)
   % ...
\end{tikzpicture}
\end{lstlisting}
\end{column}
\begin{column}{0.4\textwidth}  %%<--- here
  \begin{figure}[h]
  \centering
    \includegraphics[scale=1]{triangle_inscribedcircle.pdf}
  \end{figure}
\end{column}
\end{columns}
\end{frame}

%% ==============New Frame ================================================
\begin{frame}[fragile]
\frametitle{Euler circle}
% \vspace{-.3in}
\begin{figure}
\centering
\includegraphics{triangle_euler.pdf}
\end{figure}
\end{frame}
% subsection drawing_triangles (end)
\end{document}